\documentclass[11pt]{article}
\usepackage{amsmath,amssymb,hyperref,geometry}
\geometry{margin=1in}

\title{Towards a Communal DSL for a Universal Reasoning Substrate}
\author{Eric Robert Lawson}
\date{\today}

\begin{document}
\maketitle

\begin{abstract}
This paper proposes a framework for communal development of a domain-specific language (DSL) that operationalizes the Universal Reasoning Substrate (URST). We define a reasoning space of DSL candidates and describe an iterative refinement process using the Reasoning Axiom–Reward Feedback Loop (RARFL). Contributors can experiment, propose primitives, and validate operational semantics, creating a collaborative path toward a fully functional reasoning substrate. Meta-level reasoning, pruning, or selective exploration is handled externally, keeping the DSL lightweight and adaptable.
\end{abstract}

\section{Introduction}
The Universal Reasoning Substrate (URST) formalizes reasoning as first-class, composable objects. URST primitives include:
\begin{itemize}
    \item Objectifying reason (RDUs)
    \item Initial operational lens (path travel)
    \item Context integration
    \item Reasoning space assimilation
    \item Reward function process (RARFL)
    \item Emergent derivative reasoning space
\end{itemize}

Current general-purpose languages are insufficient to express, compose, and optimize these primitives efficiently. A dedicated DSL is required to provide a substrate for reasoning experimentation.

\section{Reasoning Space of DSL Candidates}
We define a \textit{reasoning space} $\mathcal{L}$, where each node represents a candidate language primitive, and edges represent composable interactions:
\[
\mathcal{L} = \{ \text{primitive}_i \,|\, i \in 1..N \}, \quad E = \{ (\text{primitive}_i, \text{primitive}_j) \,|\, \text{composable} \}
\]

Properties for primitives include:
\begin{itemize}
    \item Composability
    \item Context propagation
    \item Compute-once caching
    \item Extensibility for external meta-operators
\end{itemize}

\section{Iterative Refinement via RARFL}
The RARFL loop evaluates DSL candidates and reasoning objects:
\begin{enumerate}
    \item Propose new primitives or compositions.
    \item Evaluate via reward signals (axiom stability, efficiency, cross-domain transfer).
    \item Assimilate effective candidates into the reasoning space.
    \item Generate emergent derivative reasoning spaces.
    \item Repeat iteratively.
\end{enumerate}

\section{Communal Contribution Model}
Contributors can:
\begin{itemize}
    \item Propose new primitives or operational rules.
    \item Validate candidates with automated tests or small domains.
    \item Merge validated contributions via the reasoning-space DAG.
    \item Implement external meta-level reasoning loops (pruning, reward evaluation, selective exploration) and integrate insights back into the communal reasoning space.
\end{itemize}

\section{Segmentation \& Self-Reference}
To handle potentially infinite reasoning spaces, we use finite segmentation:
\begin{itemize}
    \item Example: GPS navigation segments a continuous world into roads, reducing combinatorial complexity.
    \item DSL primitives can operate within bounded subspaces while preserving universal reasoning properties.
    \item External meta-level tools (e.g., IDE plugins, ML-guided exploration) can prune or focus reasoning subspaces without modifying the core DSL.
\end{itemize}

\section{Experimental Plan}
A simple experimental domain (Tic-Tac-Toe):
\begin{enumerate}
    \item Construct a complete RDU DAG of all game states.
    \item Apply RARFL externally to derive axioms (optimal strategies).
    \item Track compute-once reuse to measure efficiency.
    \item Feed insights back into DSL refinement.
    \item Optional: external tools can selectively prune, explore, or highlight subspaces for focused experimentation.
\end{enumerate}

\subsection{DSL Primitive Example}
The DSL provides a substrate for defining and manipulating reasoning objects (RDUs) and their relationships. It does **not** itself implement meta-reasoning loops (RARFL), pruning, or selective exploration; these operations are handled externally. This design allows multiple contributors to experiment with different meta-strategies locally and share results back into the communal reasoning space, maintaining flexibility and modularity.

\begin{verbatim}
# Example pseudo-DSL in Python-embedded form
node("state", value=board_state)              # Define a reasoning state
layer("action", func=apply_move)             # Define an action layer that transforms states
pot_generator("next_states", context="board", meta="compute_once")  # Generate candidate next states efficiently
# External meta-level operations (pruning, reward evaluation, exploration heuristics) interact with the reasoning space
\end{verbatim}

\textbf{Key points:}
\begin{itemize}
    \item The DSL defines objects, transformations, and composable primitives that form a flexible reasoning substrate.
    \item Compute-once caching is embedded to optimize object reuse without constraining experimentation.
    \item Meta-reasoning, pruning, selective exploration, or ML-guided optimization are handled externally and can be integrated back into the reasoning space.
    \item This separation allows the DSL to remain lightweight, modular, and communal, while supporting flexible experimentation necessary for operationalizing a universal reasoning substrate.
    \item Contributors can adapt their meta-level strategies without altering the core DSL, enabling diverse approaches and incremental progress toward more powerful reasoning operations.
\end{itemize}

\section{Conclusion \& Next Steps}
This framework enables communal iterative refinement of a DSL to operationalize the URST. External meta-level tools and contributors can focus, prune, or optimize reasoning experiments while the DSL remains a stable, lightweight substrate. This modular approach fosters collaborative experimentation toward a fully functional universal reasoning substrate.

\end{document}
