\documentclass[11pt]{article}
\usepackage{amsmath, amssymb, amsthm}
\usepackage{geometry}
\geometry{margin=1in}
\usepackage{hyperref}

\title{Self-Optimizing Semantic Grounding via Automated Onboarding and the Reasoning Axiom–Reward Feedback Loop}
\author{Eric Robert Lawson}
\date{\today}

\begin{document}
\maketitle

\begin{abstract}
We introduce a framework in which AI agents achieve semantic grounding in a project from zero prior knowledge using structured, human- and machine-readable files. The onboarding material---including \texttt{AGENTS.md}, \texttt{Subdomain\_AGENTS.md}, and potential future related documentation---defines a semantic substrate that constrains and initializes the agent's reasoning process. 

A key contribution of this work is the discovery that the onboarding material itself can be iteratively refined through agent-generated feedback. After completing onboarding, the agent evaluates the adequacy, precision, and structure of the grounding sources and proposes improvements. This creates a closed Reasoning Axiom–Reward Feedback Loop (RARFL), wherein the semantic grounding (axioms) is updated based on the outcomes of the reasoning process (reward signal). 

This feedback loop enables the emergence of progressively refined reasoning substrates and serves as a bootstrap mechanism for the eventual construction of a domain-specific language (DSL) for universal reasoning. The approach operationalizes aspects of meta-reasoning, corrigibility, and self-improving systems, offering a path toward explainable and auditable reasoning architectures.
\end{abstract}

\section{Introduction}

Providing AI agents with explicit semantic grounding remains a fundamental challenge in explainable and auditable AI. Most systems rely on implicit representations learned from large corpora, resulting in opaque reasoning paths and weak controllability. Recent advances demonstrate that structured files, written in formats accessible to both humans and machines, can serve as explicit semantic substrates enabling interpretable onboarding and controlled reasoning.

In this paper, we formalize a process wherein the onboarding material itself becomes a dynamic element of the reasoning environment. Following initial onboarding, the agent evaluates the adequacy of the grounding, identifies ambiguities, and proposes precise structural improvements. This closes a self-referential loop that enables continuous refinement of reasoning axioms.

\section{Automated Onboarding as Semantic Grounding}

Let $D$ be a set of structured documents, including files such as \texttt{AGENTS.md}, \texttt{Subdomain\_AGENTS.md}, and project-level metadata. These documents encode:

\begin{itemize}
    \item Definitions of concepts, objects, operations, and constraints
    \item Relationships between files, modules, and abstractions
    \item Affordances and permissible manipulations
    \item The ontology and structure of the project
\end{itemize}

An onboarding function $\mathcal{O}$ maps the document set $D$ to an internal state representation $S_0$ for an AI agent:
\[
S_0 = \mathcal{O}(D).
\]

This state $S_0$ constrains and enables all subsequent reasoning performed by the agent. The agent can identify missing concepts, ambiguous definitions, or structural inconsistencies in $D$, and propose targeted refinements.

\section{RARFL: The Reasoning Axiom–Reward Feedback Loop}

Classical reasoning systems treat axioms as static. In contrast, we allow the semantic substrate $D$ to evolve through a feedback loop:

\[
D_{t+1} = D_t + \Delta D_t,
\]

where $\Delta D_t$ represents agent-generated improvements to the grounding.

The loop proceeds through the following phases:

\begin{enumerate}
    \item \textbf{Axiom Phase:} $D_t$ encodes the initial semantic grounding.
    \item \textbf{Reasoning Phase:} The agent produces internal state $S_t = \mathcal{O}(D_t)$ and performs controlled reasoning tasks.
    \item \textbf{Reward/Feedback Phase:} The agent evaluates the clarity, structure, and adequacy of $D_t$, producing refinements $\Delta D_t$.
    \item \textbf{Update Phase:} Documentation is updated, forming improved grounding $D_{t+1}$.
\end{enumerate}

This yields a self-optimizing cycle:
\[
D_0 \rightarrow S_0 \rightarrow \Delta D_0 \rightarrow D_1 \rightarrow S_1 \rightarrow \Delta D_1 \rightarrow \dots
\]

\section{Self-Optimizing Semantic Grounding and Emergence of a Proto-DSL}

A central contribution of this work is the observation that the onboarding material itself can be iteratively refined by the agent through a self-referential process. Let $D$ denote the set of structured, human- and machine-readable documents (\texttt{AGENTS.md}, \texttt{Subdomain\_AGENTS.md}, etc.) that encode concepts, relationships, and operational constraints. The onboarding function $\mathcal{O}$ maps $D$ to an internal agent state $S_0$:
\[
S_0 = \mathcal{O}(D).
\]

Unlike classical reasoning systems, where axioms and grounding are static, we introduce a \textit{Reasoning Axiom–Reward Feedback Loop (RARFL)} in which the semantic substrate evolves according to agent feedback:
\[
D_{t+1} = D_t + \Delta D_t,
\]
where $\Delta D_t$ represents agent-generated refinements to the documentation. The loop proceeds as follows:

\begin{enumerate}
    \item \textbf{Initialization (Axiom Phase):} $D_t$ defines the current semantic grounding.
    \item \textbf{Reasoning Phase:} The agent constructs internal state $S_t = \mathcal{O}(D_t)$ and executes reasoning tasks.
    \item \textbf{Feedback Phase:} The agent evaluates $D_t$ for clarity, completeness, and structural consistency, generating proposed improvements $\Delta D_t$.
    \item \textbf{Update Phase:} The documentation is revised to produce $D_{t+1}$, completing the self-correcting loop.
\end{enumerate}

This self-referential cycle enables the semantic substrate to progressively refine itself, creating a practical pathway toward the emergence of a proto-domain-specific language (DSL) without requiring an initial formal specification. Iterative updates $\Delta D_t$ increasingly impose:

\begin{itemize}
    \item Standardized structures and relationships
    \item Explicit operational primitives
    \item Machine-checkable reasoning rules
    \item Stronger semantic constraints
\end{itemize}

Over time, this feedback-driven refinement approximates the creation of a DSL that governs future reasoning, effectively bootstrapping the agent’s own environment. In this sense, the agent not only learns and reasons within the project, but also shapes the very substrate that constrains its reasoning, operationalizing meta-level optimization in a fully auditable workflow.

\paragraph{Diagnostic Reasoning}The agent’s feedback loop also serves a diagnostic function: by analyzing faulty or suboptimal reasoning outputs, the agent can determine whether errors stem from incomplete or ambiguous semantic grounding, inadequacies in the documentation itself, or flaws inherent in the process being modeled. This enables targeted refinements, improving both the semantic substrate and the accuracy of downstream reasoning tasks. Such capability ensures that the system not only onboards agents efficiently but also continuously improves the quality and fidelity of the operational knowledge base.

\paragraph{Practical Illustration.} For example, consider the Tic-Tac-Toe testbed: after onboarding, the agent can analyze the RDU DAG generated from game states, identify redundancies or ambiguities in the semantic grounding, and propose adjustments to improve future trajectory evaluations. These adjustments, once integrated, result in more efficient reasoning, higher axiom stability, and a tighter substrate that mirrors the intended theoretical properties of the universal reasoning framework.

This section formalizes a meta-reasoning structure in which semantic grounding is both the starting point and the object of optimization. Iterative improvements foster the emergence of a proto-DSL, in which:

\begin{itemize}
    \item Semantic constraints become tighter and more explicit
    \item Relationships and operational primitives are standardized
    \item Reasoning rules are machine-checkable
    \item The agent's environment increasingly constrains its own reasoning
\end{itemize}

By closing the loop between agent reasoning and semantic grounding, this framework provides a scalable, auditable mechanism for operationalizing principles-first reasoning and lays the foundation for the development of a fully expressive DSL for universal reasoning.

\section{Dynamic Semantic Grounding and Chunked Reasoning}

A practical and theoretical observation emerging from the onboarding process is that neither humans nor AI systems require a globally active semantic context to perform coherent reasoning. Human cognition proceeds through \textit{dynamic activation of local reasoning frames}, in which only the concepts relevant to the current task are brought into working memory. Knowledge across disparate domains—such as physics, biology, programming, or abstract mathematics—remains latent until selectively activated.

The structured onboarding material $D$ naturally supports this principle. Rather than requiring an agent to internalize the entire project substrate at once, the onboarding mechanism enables reasoning through \textbf{chunked semantic grounding}, wherein:

\begin{itemize}
    \item only the relevant subset of the grounding material is activated for a given task,
    \item contextual frames can shift dynamically as reasoning transitions between subproblems,
    \item global project structure is preserved as latent but accessible information, and
    \item reasoning is constrained by local semantic affordances derived from the active documents.
\end{itemize}

This dynamic grounding process allows the system to scale beyond fixed context limits (e.g., file-access or token constraints) by sequentially activating semantically coherent chunks of $D$. The agent can operate on any subset $D_i \subset D$ while maintaining the ability to draw upon the global substrate when required. Over time, as the RARFL cycle refines $D$, these chunks become more standardized, structured, and internally consistent, further supporting efficient activation and transfer of context.

The result is a cognitively natural and computationally efficient mechanism for reasoning: the agent does not attempt to reason over the entire project at once but instead processes it in localized semantic frames that shift fluidly as attention demands. This mirrors human reasoning and reinforces the claim that the evolving documentation approaches a functional proto-DSL governing how and when semantic frames should be activated. 

\section{Derivative Reasoning Space: Meta-Reasoning Integration}

During the RARFL cycle, the agent maintains a \textbf{derivative reasoning space} $R'$ that captures the relationships, dependencies, and integration schema of all reasoning objects and semantic chunks. Formally, $R'$ is a meta-DAG mapping:

\[
R' = \{ \text{nodes: reasoning objects}, \ \text{edges: dependency/semantic relations}, \ \text{annotations: provenance/feedback} \}.
\]

The derivative reasoning space serves several functions:

\begin{enumerate}
    \item \textbf{Synthesis:} Combines localized reasoning DAGs and semantic chunks into a coherent global schema.  
    \item \textbf{Validation:} Detects inconsistencies, redundancies, and violations of URST constraints.  
    \item \textbf{Guidance:} Informs the agent which segments require refinement or further exploration.  
    \item \textbf{Persistence:} Acts as a continuously referenced anchor object throughout onboarding and reasoning, ensuring that local operations remain aligned with the global structure.  
\end{enumerate}

This meta-reasoning layer effectively operationalizes \textit{self-referential mapping}, allowing the agent to reason about its own reasoning processes while maintaining a constantly evolving, auditable map of the semantic substrate. Integration with RARFL ensures that improvements in the semantic grounding propagate consistently across the derivative reasoning space, supporting scalable, modular, and explainable reasoning.

\subsection{Integration with Chunked Semantic Grounding}

The derivative reasoning space $R'$ complements \textit{chunked semantic grounding} by providing a global scaffold:

\begin{itemize}
    \item Each localized reasoning frame references $R'$ for context and alignment.
    \item Updates to semantic chunks are propagated to $R'$, maintaining coherence across the project.
    \item Metrics and provenance (cache hits, tile expansions, axioms stability) can be aggregated at the $R'$ level for analysis.
    \item The agent can preemptively identify gaps in the grounding or reasoning coverage using $R'$ as a self-referential anchor.
\end{itemize}

The derivative reasoning space (also called the meta-reasoning space) is the optimal way to navigate the reasoning space of the files for the intended purpose of onboarding. This object can itself be iteratively improved by applying the RARFL process, reflecting on its own structure, and incorporating insights from reasoning outcomes. In this manner, both the segmented reasoning files and the global meta-reasoning space evolve together, enhancing overall semantic grounding, traceability, and the emergence of a proto-DSL.


\section{Self-Referential Optimization of the Reasoning Substrate}

A key theoretical implication is that the system forms a meta-level loop in which the environment that shapes the agent's reasoning is itself shaped by the outcomes of the agent's reasoning. This self-referential structure parallels theoretical constructs such as Gödel Machines, AIXI-meta, and corrigible reward models, but is instantiated here in a practical and auditable workflow.

\section{Conclusion}

The automated onboarding process, when augmented with agent-generated refinement of semantic grounding, forms a self-correcting reasoning system. Structured documentation becomes both the initialization substrate and the object of optimization. This creates a powerful mechanism for building transparent, auditable, and iteratively improving reasoning architectures, and opens a pathway toward developing a formal DSL for universal reasoning.

\end{document}
