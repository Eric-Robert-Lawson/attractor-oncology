\documentclass[11pt]{article}
\usepackage{amsmath, amssymb, amsthm}
\usepackage{geometry}
\usepackage{hyperref}
\geometry{margin=1in}

\newtheorem{definition}{Definition}[section]
\newtheorem{remark}{Remark}[section]

\title{Causal Equivalence and Compatibility:\\
A Subfield of the Universal Reasoning Substrate Theory (URST)}
\author{Eric Robert Lawson}
\date{\today}

\begin{document}

\maketitle

\begin{abstract}
This document introduces \textbf{Causal Equivalence and Compatibility} as a new subfield emerging from the Universal Reasoning Substrate Theory (URST). By defining reasoning spaces, derivative reasoning spaces, and RARFL as foundational constructs, we formalize a framework to measure and analyze causal structure across domains. Causal equivalence provides a continuous, measurable, and explainable substrate, while causal compatibility defines the structural preconditions necessary for equivalence to exist. This subfield enables precise cross-domain reasoning transfer, emergent optimization, and explainable causal analysis.
\end{abstract}

\section{Introduction}
The Universal Reasoning Substrate Theory (URST) provides a framework for modeling reasoning as a manipulable and objectifiable substrate. Within URST, reasoning spaces can be formalized, optimized, and transferred across domains via derivative reasoning spaces and the RARFL process.  

A natural corollary of URST is the concept of \textbf{causal equivalence}, which measures the degree to which reasoning trajectories across distinct domains produce equivalent causal outcomes. Closely related is \textbf{causal compatibility}, which defines the structural preconditions for equivalence to exist.

\section{Reasoning Spaces}
\begin{definition}[Reasoning Space]
A reasoning space $\mathcal{R}$ is a structured set of reasoning trajectories, where each trajectory represents a sequence of steps taken by an agent to achieve an objective under a specified reward function.
\end{definition}

\begin{definition}[Derivative Reasoning Space]
Given a reasoning space $\mathcal{R}$, a derivative reasoning space $\mathcal{R}'$ is a transformed space that encodes optimized or partially solved trajectories of $\mathcal{R}$, typically via RARFL or other reasoning optimization methods.
\end{definition}

\section{Causal Equivalence}
\begin{definition}[Causal Equivalence]
Two reasoning spaces $\mathcal{R}_1$ and $\mathcal{R}_2$ are \textbf{causally equivalent} over a subset of trajectories $S \subseteq \mathcal{R}_1 \times \mathcal{R}_2$ if the outcomes under corresponding trajectories produce equivalent effects relative to their respective reward functions.  

Formally, define a metric $\delta$ measuring divergence:
\[
\delta((\tau_1, \tau_2)) = \text{difference in causal effect of trajectories } \tau_1 \in \mathcal{R}_1, \tau_2 \in \mathcal{R}_2.
\]

Causal equivalence is satisfied when $\delta(\tau_1, \tau_2) \leq \epsilon$ for all $(\tau_1, \tau_2) \in S$, where $\epsilon$ is a threshold for acceptable divergence.
\end{definition}

\section{Causal Compatibility}
\begin{definition}[Causal Compatibility]
Two reasoning spaces $\mathcal{R}_1$ and $\mathcal{R}_2$ are \textbf{causally compatible} if there exists a non-empty subset of trajectories $S \subseteq \mathcal{R}_1 \times \mathcal{R}_2$ over which causal equivalence can be measured.  

If the reasoning spaces are incompatible, no mapping exists and causal equivalence is undefined.
\end{definition}

\begin{remark}
Causal compatibility defines the structural preconditions for equivalence. It distinguishes whether reasoning spaces are even comparable, while causal equivalence measures the degree of alignment where comparability exists.
\end{remark}

\section{Implications and Applications}

\begin{itemize}

    \item \textbf{Context-Sensitive Cross-Domain Transfer:}  
    Causal equivalence $\delta$ is interpreted as a relativistic, context-dependent measure, influenced by the semantic grounding of the target reasoning space. Transfer decisions should consider not only the raw divergence but also the semantic efficiency $\eta_i$ of each reasoning segment, enabling priority weighting toward high-value segments. Thresholds $\epsilon$ may dynamically adapt across tiles or segments, supporting nuanced assimilation of knowledge from one domain to another.

    \item \textbf{Segment-Level Cognitive Fitting:}  
    At advanced reasoning stages, equivalence is assessed not globally but over discrete reasoning segments or hierarchical tiles. This allows emergent relationships to be identified between local structures of reasoning spaces, facilitating precise, context-aware transfer that respects semantic coherence and avoids unsafe or incoherent assimilation.

    \item \textbf{Explainable AI and Divergence Profiling:}  
    Divergence hotspots where $\delta > \epsilon$ highlight specific regions or segments where causal reasoning fails to transfer. These provide interpretable explanations for reasoning mismatches and inform corrective strategies, including pruning, local refinement, or targeted RARFL interventions.

    \item \textbf{Emergent Optimization via RARFL:}  
    Causal equivalence can be integrated into RARFL reward loops, where candidate axioms or strategies from one domain are evaluated for their projected effect in a target space. Weighted $\delta$ values guide conservative or aggressive assimilation, balancing coherence gain ($\Delta C$) against semantic grounding cost ($\Delta S$) and segment-level efficiency $\eta_i$.

    \item \textbf{Operational DSL Integration:}  
    - DSL primitives should expose:
        \begin{itemize}
            \item \texttt{register\_mapping(source, target, mapping\_fn)}: stores alignment functions for trajectory mapping.
            \item \texttt{compute\_delta(source\_traj, target\_traj, method)}: computes $\delta$ using staged proxies (outcome, structural, intervention, embedding).
            \item \texttt{declare\_compatibility(source, target, $\epsilon$)}: sets dynamic divergence thresholds.
            \item \texttt{annotate\_metaRDU(metaRDU\_id, $\delta$, method, provenance)}: stores $\delta$ and mapping metadata for auditability.
        \end{itemize}
    - Machine agents should leverage $\delta$ to prioritize tile expansions, refine meta-RDUs, and trigger context-aware RARFL corrections.

    \item \textbf{Foundational Theory for ASI-Stage Transfer:}  
    By grounding $\delta$ in segment-level semantics and emergent inter-object relations, this framework represents a critical stage of advanced reasoning: leveraging highly-understood narrow reasoning spaces to accelerate mapping, learning, and optimization of broader or novel domains. Causal equivalence thus operationalizes the final stage of cognitive fitting, where reasoning spaces are actively compared, aligned, and safely transferred in a relativistic, semantic-aware manner.

\end{itemize}


\section{Conclusion}
Causal Equivalence and Compatibility, as a subfield of URST, provides a rigorous foundation for reasoning about causal relationships across domains. By shifting the perspective from binary inference to measurable equivalence, and introducing structural compatibility as a precondition, this subfield opens pathways for cross-domain reasoning transfer, emergent optimization, and explainable AI.  

This framework represents a corollary epiphany from URST, extending the theoretical reach of reasoning-space analysis and establishing a new, operationally measurable field for causality.

\end{document}
