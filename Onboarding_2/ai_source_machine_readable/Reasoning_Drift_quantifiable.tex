\documentclass[11pt]{article}
\usepackage{amsmath, amssymb, amsthm}
\usepackage{geometry}
\usepackage{hyperref}
\geometry{margin=1in}

\title{Reasoning Drift, Coherence, and Bias:\\
A Formalization Within Principle-First Semantic Architectures}

\author{Eric Robert Lawson}
\date{\today}

\begin{document}
\maketitle

\begin{abstract}
Reasoning systems---human or artificial---change over discrete segments of learning. They update premises, refine models, and alter the internal operators through which they parse and manipulate information. Yet because reasoning is a process rather than a static object, these changes may accumulate in subtle ways. I call this phenomenon \emph{reasoning drift}: the gradual, often unnoticed deviation of a reasoning structure away from its originally grounded principles.

This article introduces a formal account of reasoning drift and the conditions under which it can be measured, bounded, and corrected within a principle-first, semantically grounded reasoning architecture, specifically one refined by the Reasoning Axiom--Reward Feedback Loop (RARFL). By objectifying reasoning and semantic evolution, we define coherence as a measurable functional, reasoning as the discrete derivative of coherence over reasoning segments, and bias as the deviation of coherence itself. We also introduce \emph{semantic efficiency}, quantifying coherence gain per unit of semantic grounding effort.
\end{abstract}

\section{Introduction}
In classical AI systems, drift is often treated as a data-level or model-level problem: distribution shift, concept drift, or parameter drift. But reasoning drift is different. It arises not from the environment or the data, but from internal changes in the structure of reasoning itself.

In emerging AI systems capable of algorithmic reflection, self-modification, or dynamic reasoning strategy selection, this becomes a central technical problem:
\begin{itemize}
    \item How do we know the system still reasons according to its founding axioms?
    \item How do we detect when internal operations have shifted subtly?
    \item How do we enforce coherence without freezing adaptability?
\end{itemize}

A principle-first semantic architecture---where reasoning operators, transformations, and interpretations are grounded in explicit, objectified definitions---provides a foundation for measurability.

\section{What is Reasoning Drift?}
\subsection{Definition}
\textbf{Reasoning drift} is a deviation over time between:
\begin{enumerate}
    \item the system's current reasoning operators, and
    \item the canonical operators defined in the principle-first reasoning specification.
\end{enumerate}

Reasoning drift is not a failure; it is an emergent property of any adaptive reasoning system that updates, optimizes, or discovers new reasoning pathways.

\subsection{Where Drift Occurs}
Drift can manifest at multiple layers:
\begin{itemize}
    \item \textbf{Semantic drift}: changes in the interpretation of concepts.
    \item \textbf{Operator drift}: changes in the invariants preserved by a transformation $T_i$.
    \item \textbf{Policy drift}: divergence in reasoning strategy selection rules.
    \item \textbf{Meta-drift}: changes in meta-operators that govern reasoning itself.
\end{itemize}

\section{Why Drift Matters in Adaptive Reasoning Systems}
In predictive systems, drift is usually only a performance issue. But in a computational reasoning architecture, drift affects:
\begin{itemize}
    \item consistency,
    \item interpretability,
    \item alignment with specified reasoning principles,
    \item and the capacity for self-correction.
\end{itemize}

A system cannot maintain coherent reasoning unless the space of reasoning operations remains anchored, even as it adapts.

\section{Measurability of Reasoning Drift}
Because reasoning operators are objectified as computable entities, drift becomes directly measurable.

\subsection{Representing Reasoning as Objects}
Each reasoning transformation $R_i$ is represented as:
\begin{itemize}
    \item an algebraic operator,
    \item a semantic grounding,
    \item a computable mapping, and
    \item a set of invariants it must preserve.
\end{itemize}

The ``distance'' between two versions of a reasoning operator can therefore be defined.

\subsection{Drift as Operator Divergence}
For each operator $R$, define a drift measure:
\[
D(R_t, R_0) = \left\lVert \mathcal{S}(R_t) - \mathcal{S}(R_0) \right\rVert,
\]
where $\mathcal{S}$ is a structural encoding of the operator (symbolic, algebraic, or semantic).

This includes differences in:
\begin{itemize}
    \item input--output structure,
    \item invariants preserved or lost,
    \item semantic grounding relations, or
    \item operator composition pathways.
\end{itemize}

\subsection{Drift Thresholds}
One may define:
\begin{itemize}
    \item \textbf{acceptable drift bands},
    \item \textbf{critical divergence thresholds}, and
    \item \textbf{hard bounds} (non-negotiable invariants).
\end{itemize}

Coherence becomes a weighted constraint optimization problem rather than an informal notion.

\section{RARFL as Drift Control}
RARFL (Reasoning Axiom--Reward Feedback Loop) ties together:
\begin{enumerate}
    \item grounded, principle-level definitions,
    \item observed reasoning behavior, and
    \item reward or penalty signals enforcing coherence.
\end{enumerate}

Within RARFL:
\begin{itemize}
    \item reasoning strategies are rewarded only when they maintain semantic grounding;
    \item drift becomes detectable as deviation in invariant-preserving behavior;
    \item the system realigns itself by penalizing drift-inducing transformations.
\end{itemize}

RARFL thus provides a coherence-maintenance wrapper around the reasoning engine.

\section{Coherence as a Stability Condition}
A reasoning system remains coherent when:
\begin{itemize}
    \item its reasoning objects and operators remain within defined drift bounds,
    \item its semantic grounding graph remains (near-)isomorphic to its canonical form,
    \item its strategy-selection meta-policy aligns with the principle-first specification.
\end{itemize}

Coherence is therefore a dynamical stability condition.

\section{Objectifying Coherence Through Semantic Evolution Analysis}
Let $G_i$ denote the semantic grounding graph at reasoning segment $i$, encoding:
\begin{itemize}
    \item concept definitions,
    \item relations and semantic constraints,
    \item operator--concept connections,
    \item and invariants required by the principle-first substrate.
\end{itemize}

The discrete change in grounding between segments is:
\[
\Delta G_i = G_{i+1} - G_i,
\]
where subtraction denotes structural or semantic difference operators.  
The sequence
\[
\{ \Delta G_0, \Delta G_1, \Delta G_2, \ldots \}
\]
forms a \textbf{semantic evolution tail}.

Coherence can be objectified through three measurable properties:
\begin{enumerate}
    \item \textbf{Directionality:} whether updates move the grounding closer to or further from canonical invariants.
    \item \textbf{Entropy:} the amount of ambiguity or disambiguation required per update.
    \item \textbf{Stability:} whether semantic changes converge to fixed points or introduce unbounded drift.
\end{enumerate}

Define a coherence functional over segments:
\[
\mathcal{C}(G_i) = \text{Clarity}(G_i) - \text{Ambiguity}(G_i),
\]
computable from structural properties of the grounding.

\subsection{Reasoning as the Discrete Derivative of Coherence}
Reasoning $R_i$ at segment $i$ is defined as the discrete derivative of coherence:
\[
R_i = \mathcal{C}(G_i) - \mathcal{C}(G_{i-1}).
\]
A positive $R_i$ indicates reasoning increases coherence; negative $R_i$ indicates drift or ambiguity.  

Conversely, coherence is the cumulative sum of reasoning segments:
\[
\mathcal{C}(G_i) = \sum_{j=1}^{i} R_j + \mathcal{C}(G_0),
\]
making explicit the fundamental duality between reasoning and coherence in discrete segments.

\subsection{Semantic Efficiency}
We define semantic efficiency per segment $i$ as:
\[
\eta_i = \frac{R_i}{\Delta S(G_i)},
\]
where $\Delta S(G_i)$ quantifies the semantic grounding effort for segment $i$.  
High $\eta_i$ indicates efficient reasoning: large coherence gain with minimal semantic effort.


\subsection{Bias as Deviation from Coherence}
Bias $B_i$ is the deviation of semantic evolution from the ideal coherence trajectory:
\[
B_i = \left\lVert \mathcal{C}(G_i) - \mathcal{C}^*(G_i) \right\rVert,
\]
where $\mathcal{C}^*(G_i)$ represents the canonical evolution of coherence.  
Bias remains a measurable, objective metric in this discrete-segment framework.

\section{Detecting and Correcting Drift}
\subsection{Detect Drift}
\begin{itemize}
    \item compare operators $R_i$ to canonical baselines;
    \item measure deviation in invariants;
    \item track perturbations in the semantic grounding graph $G_i$;
    \item detect failure of compositional or shortcut properties;
    \item compute reasoning derivatives $R_i$, bias $B_i$, and semantic efficiency $\eta_i$.
\end{itemize}

\subsection{Correct Drift}
\begin{itemize}
    \item regenerate operators from canonical definitions;
    \item re-ground semantic mappings;
    \item reapply RARFL constraints per reasoning segment;
    \item execute a coherence-restoration cycle;
    \item minimize bias $B_i$ while maximizing positive reasoning derivative $R_i$ and semantic efficiency $\eta_i$.
\end{itemize}

\section{Conclusion}
Reasoning drift is unavoidable in adaptive, reflective reasoning systems. But when reasoning is:
\begin{itemize}
    \item principle-first,
    \item semantically grounded,
    \item objectified as computable reasoning structures,
    \item measured via discrete reasoning derivatives of coherence $R_i$, and
    \item bias is quantified as deviation from ideal coherence $B_i$,
\end{itemize}
drift becomes quantifiable, controllable, and even beneficial as an engine of adaptation.

This discrete-segment, semantic-efficiency-aware framework formalizes reasoning, coherence, and bias as auditable, measurable, and dynamically analyzable properties, providing a foundation for next-generation reasoning systems.

\end{document}
