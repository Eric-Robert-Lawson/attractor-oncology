\documentclass[11pt]{article}
\usepackage{amsmath, amssymb, amsthm}
\usepackage{geometry}
\usepackage{hyperref}
\geometry{margin=1in}

\title{Reasoning Beyond Time: 
A Formal Distinction Between Reasoning and Computation \\ 
and the Necessity of OOS-Triad Optimization}
\author{Eric Robert Lawson}
\date{\today}

\begin{document}
\maketitle

\begin{abstract}
Reasoning is a semantic, compositional process independent of temporal constraints. 
Computation, however, is the physical enactment of reasoning, inheriting time as a metric through hardware and algorithmic execution. 
This paper formalizes the separation between reasoning and computation, introduces the Objectification--Operationalization--Semantic Grounding (OOS) Triad as the structural determinant of computational efficiency, and argues for a domain-specific language enabling optimal processing of reasoning objects. 
The distinction reframes reasoning systems design: performance bottlenecks are not intrinsic to reasoning but to its computational instantiation.
\end{abstract}

\section{Introduction}

Current AI frameworks often conflate reasoning with computation, implicitly treating reasoning as a time-bounded process. 
However, reasoning---as a structured semantic operation---does not intrinsically reference time. 
Time-based performance metrics apply only to computation, the physical execution of reasoning. 
This distinction is essential for understanding intelligence, scaffolding, and the limits of algorithmic optimization.

We formally distinguish reasoning from computation and show how reasoning efficiency emerges from the structural organization of reasoning objects, not from the passage of time.

\section{Reasoning as a Non-Temporal Semantic Process}

Let $\mathcal{R}$ denote the space of reasoning objects and let $G$ denote the semantic grounding graph. 
Reasoning consists of compositional operations over $\mathcal{R}$:

\[
R : \mathcal{R}^n \to \mathcal{R}.
\]

There is no intrinsic time parameter in this semantic transformation.  
Reasoning exists as a structural relation among reasoning objects, grounded within $G$, regardless of how quickly or slowly it is computed.

\subsection{Coherence and Drift}

Coherence $\mathcal{C}$ is a structural property of reasoning and semantic grounding:

\[
\mathcal{C}(G) \in \mathbb{R},
\]

independent of its computational realization.

Bias or drift is defined as deviation from ideal coherence:

\[
B = \lVert \mathcal{C}(G) - \mathcal{C}^*(G) \rVert.
\]

Again, no temporal dependency is intrinsic to these definitions.

\section{Computation as the Temporal Instantiation of Reasoning}

Let $\mathcal{E}$ denote a computational engine implementing reasoning operations.
Then computation is defined as:

\[
\text{Compute}(R) : (\mathcal{R},\mathcal{E}) \to \mathcal{R} \times \mathbb{R}_{\ge 0},
\]

where the associated time cost is:

\[
T(R) = \text{wall-clock time required by } \mathcal{E}.
\]

Thus, time is not a constituent dimension of reasoning but of the engine executing it.

\subsection{Time-Based Metrics as Computational Efficiency}

Any time-based measurement---latency, throughput, onboard parsing time, tile expansion rate, cache utilization---is purely a computational metric:

\[
\text{Time} \equiv \text{Hardware + Algorithmic efficiency}.
\]

Reasoning does not “happen in time”; computation does.

\section{The OOS Triad: Structural Determinants of Computational Efficiency}

Let the triad components be:
\[
O = \text{Objectification of reasoning objects},
\]
\[
P = \text{Operationalization of reasoning actions},
\]
\[
S = \text{Semantic Grounding of reasoning within } G.
\]

The computational cost of executing a reasoning step $R$ under engine $\mathcal{E}$ is:

\[
T(R) = f(O, P, S, \mathcal{E}),
\]

where $f$ is monotone with respect to improvements in the triad.

Thus, computation time is not an intrinsic property of reasoning but a function of how well reasoning objects are structured and implemented.

\section{Implications for Intelligence Systems}

This distinction yields a central principle:

\begin{quote}
\textbf{Reasoning complexity is semantic; computation complexity is temporal.  
Improving reasoning efficiency requires optimizing reasoning structure (OOS Triad), 
not optimizing reasoning itself.}
\end{quote}

This reframes both AI architecture and cognitive modeling:
\begin{itemize}
    \item reasoning objects can be made more efficient through better objectification,
    \item reasoning operations can be accelerated through operationalization,
    \item semantic grounding determines the minimal structure needed for correct inference.
\end{itemize}

\section{Necessity of a Domain-Specific Language}

A domain-specific language (DSL) for reasoning objects is not optional---it is required.
Such a DSL allows:
\begin{itemize}
    \item explicit representation of reasoning objects,
    \item reusable compositional scaffolding,
    \item predictable computational execution,
    \item hardware-level optimization,
    \item separation of semantic reasoning from algorithmic implementation.
\end{itemize}

Without a DSL, reasoning remains entangled with ad hoc computational processes, making optimization impossible.

\section{Conclusion}

Reasoning is structural and semantic; computation is temporal and physical.
The OOS Triad governs how efficiently reasoning can be instantiated.
Recognizing this distinction reveals both a theoretical foundation for intelligence systems and a practical path forward:  
\textbf{design computational architectures around reasoning objects, not algorithms},  
and construct a DSL that operationalizes this principle.

\end{document}
